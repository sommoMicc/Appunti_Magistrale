% !TEX encoding = UTF-8
% !TEX TS-program = pdflatex
% !TEX root = ../apprendimento_automatico.tex
% !TEX spellcheck = it-IT
\section{Apprendimento automatico}\label{apprendimento-automatico}
\subsection{Descrizione del corso}
L'esame è a domande aperte. Durante l'anno vengono proposti degli esercizi per acquisire "punti esperienza", che servono a skippare il progetto.
In caso, il progetto è cumulabile ai punti esperienza. Siccome la parte pratica (punti esperienza/progetto) vale 8 punti, l'esame da al massimo 22 punti.
\\ Studiare python che serve!!

\subsection{Introduzione}
Il machine learning utilizza un ragionamento induttivo.
\subsubsection{Evoluzione del ragionamento}
Il primo tipo di ragionamento (dai tempi di aristotele) è chiamato ragionamento deduttivo. L'esempio più comune è:
\begin{itemize}
  \item Tutti gli uomini sono mortali
  \item Socrate è un uomo
\end{itemize}
Quindi Socrate è mortale!
\\
Questo tipo di ragionamento è utilizzato da gran parte delle dimostrazioni di teoremi matematici e anche dall'intelligenza artificiale "classica".
\\
Il secondo ragionamento è quello di tipo induttivo:
\begin{itemize}
  \item Socrate è un uomo
  \item Socrate è morto
\end{itemize}
Quindi tutti gli uomini sono mortali. La differenza è quella che qua abbiamo fatto una generalizzazione, un'astrazione in pratica. Mentre nel ragionamento
deduttivo partiamo da fatti veri e facciamo delle assunzioni di cui possiamo essere certi, nel ragionamento induttivo invece, datosiché facciamo un'astrazione,
ci potrebbe essere un'approssimazione.
\\
Terzo tipo di ragionamento è quello abduttivo
\begin{itemize}
  \item Tutti gli uomini sono mortali
  \item Socrate è morto
\end{itemize}
Quindi Socrate è un uomo.\\
A partire da una regola e dalla sua conseguenza, si cerca di risalire alla causa, usando la regola nel senso opposto. Anche in questo caso si fa un'assunzione
che non necessariamente è vera.

Per chiarezza, vedi esempi slide Dr House (che devo riportare)

\subsection{Cosa è il Machine Learning}
Il machine learning può essere sia un task di deduzione da dati (classificazione di dati) che di predizione di dati (se si presenta un dato mai visto prima, capire di che tipo è).
Nel machine learning ci sono varie definizioni. Le più importanti sono:
\begin{itemize}
  \item Il Machine Learning è il campo di
  studio che dà ai computer l’abilità di
  apprendere (a realizzare un compito)
  senza essere esplicitamente
  programmati a farlo
  \item Un programma impara da un’
  esperienza E rispetto a dei compiti T
  ottenendo una performance P, se
  quest’ultima migliora con l’esperienza E
\end{itemize}
Il machine learning lasce da dei "buchi applicativi" lasciati dagli algoritmi. Formalmente, un algoritmo è "una sequenza ordinata e finita di passi (operazioni o istruzioni) elementari che conduce a un \emph{ben determinato} risultato in un tempo finito"


Non sempre è possibile utilizzare degli algoritmi per risolvere un problema.
Per vari motivi:

\begin{itemize}
\item non sempre si può formalizzare un determinato problema
\item ci sono delle situazioni di rumore nell'input (in un immagine, dei pixel con valore di grigio random) o incertezza (non si riesce a dire se l'input appartiene ad una classe o ad un'altra) nell'output
\item risulta troppo complesso trovare una soluzione oppure sono richieste troppe risorse
\end{itemize}

Alcuni esempi sono: riconoscimento facciale, filtro anti-spam.

In questi casi gli algoritmi (sequenza finita di passi che portano ad un risultato determinato in un tempo finito) non funzionano ed è quindi preferibile fornire una soluzione ``\emph{imperfetta}''.

In apprendimento automatico si studiano i metodi per trasformare l'informazione empirica (dati del problema) in conoscenza. In generale, in ML, quando i dati sono pochi la soluzione non funziona.

Questo approccio è diventato possibile grazie al fatto che Internet ha reso disponibili molti dati.

\subsection{Le basi}\label{le-basi}

Perché il machine learning funzioni deve esserci un processo
(stocastico o deterministico) che spiega i dati che osserviamo, in modo
da riuscire a costruire un'approssimazione di tale processo che può
anche risultare imperfetta dal momento che il processo che si vuole
approssimare non è noto.

In ogni caso, deve esistere un processo (anche stocastico) che regola i dati che osserviamo, ovvero che li "spieghi", che ne dia un significato. Lo scopo del 
machine learning è quello di approssimare al meglio possibile il processo (non necessariamente uguale, ma una replica che sia utile a raggiungere l'obiettivo),
 senza dover per forza riprodurlo (come la metafora dell'uccello e dell'aereo).

\emph{Stocastico}: random a probabilità

L'obiettivo finale del machine learning è quello di definire dei criteri
da ottimizzare in modo che sia possibile andare a migliorare dei modelli
definiti su certi parametri.

Questi modelli possono essere:

\begin{itemize}
\item
  \textbf{Preditivi}: per fare previsioni sul futuro (es: filtro
  anti-spam)
\item
  \textbf{Descrittivi}: utilizzare dei dati per ottenere maggiori
  informazioni (data mining)
\end{itemize}

Esempi applicativi:

\begin{itemize}
\item
  Software OCR
\item
  Estrapolazione di dati a partire dal linguaggio naturale
\item
  Riconoscimento facciale
\item
  Giochi con informazione incompleta (Gaist? gioco con fantasmi
  rosso/blu, tedesco)
\end{itemize}

\subsection{Problemi tipici dell'apprendimento automatico}\label{problemi-tipici-dellapprendimento-automatico}

\begin{itemize}
\item
  \textbf{Classificazione binaria}: dato un input dire se appartiene ad
  una determinata classe o meno. Esempio: data una cifre dire se è uno 0
  o meno.
\item
  \textbf{Classificazione multiclasse}: dato un input lo assegno ad una
  determianta categoria. Es: identificare una cifra manoscritta.
\item
  \textbf{Regressione}: dato un insieme di valori, trovare una funzione
  che li approssimi.
\item
  \textbf{Ranking di classi} (non sarà affrontato): data una serie di
  dati, dire quali sono più rilevanti, ovvero, data una serie di
  documenti ordinarli nel modo migliore secondo una determinata
  preferenza, es: motore di ricerca.
\item
  \textbf{Novelty detection}: riconoscimento delle irregolarità a
  partire da una serie di dati. es: frode bancaria su una serie di
  transazioni, controllo degli accessi, ecc.
\item
  \textbf{Clustering}: raggruppamento di dati in modo gerarchico,
  basandosi su alcune caratteristiche che li accomunano o meno.
\item
  \textbf{Associazioni}: quello che fa Amazon con ``altri utenti hanno
  comprato''
\item
  \textbf{Reinforcement Learning}: valutazioni di strategie, quando si
  ha una serie di stati e possibili azioni, si vuole valutare la qualità
  complessiva, es: movimenti di un robot.
\end{itemize}

Esistono un po' di casi in cui il machine learning ha "mostrato il meglio di sé".
\textbf{Generative Adversarial Learning}: con il machine learning si può avere anche un comportamento "creativo". 
Ad esempio, con le GAN si è riuscito a scrivere poesie.