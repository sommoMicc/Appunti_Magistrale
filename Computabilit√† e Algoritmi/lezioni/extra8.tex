\section{Appello 2016-07-01}

Correzione di Baldan.

\subsection{Esercizio 1}

Sia $A$ un insieme ricorsivo e siano $f_1, f_2 : \mathbb{N} \rightarrow \mathbb{N}$ funzioni calcolabili. Dimostrare che è calcolabile la funzione $f : \mathbb{N} \rightarrow \mathbb{N}$ definita da

$$
f(x) = \begin{cases}
f_1(x) & x \in A \\
f_2(x) & x \notin A
\end{cases}
$$

Il risultato continua a valere se indeboliamo le ipotesi e assumiamo $A$ RE? Spiegare come si adatta la dimostrazione, in caso positivo, o fornire un controesempio in caso negativo.

\subsubsection{Soluzione}

Siano $e_1, e_2$ i programmi che calcolano le due funzioni date.

La funzione $f$ può essere calcolata con:

$$
f(x) = \Bigg( \mu w. \bigg( S\Big(e_1, x, (w)_1, (w)_2 \Big) \wedge x \in A \bigg) \vee \bigg( S\Big(e_2, x, (w)_1, (w)_2 \Big) \wedge x \notin A \bigg) \Bigg)_1
$$

questo perché l'appartenenza o meno ad $A$ è decidibile in quanto $A$ è ricorsivo.

Utilizzare 

$$
f(x) = f_1(x) \mathcal{X}_A(x) + f_2(x)\mathcal{X}_{\overline{A}}
$$

è \textbf{sbagliato} perché una delle due funzioni può \textbf{non essere totale} e quindi verrebbe prodotto un risultato sbagliato, utilizzando una funzione sempre indefinita. Se ci fosse anche l'ipotesi della totalità il discorso sarebbe stato diverso.

Se invece $A$ è ricorsivamente enumerabile, la non appartenenza ad $A$ non è più decidibile.

Per dimostrare questo conviene trovare un contro esempio.

Posso prendere $A=K$, $f_1(x) = 1, f_2(x) = 0$, così facendo $f$ risulta essere:

$$
f(x) = \begin{cases}
1 & x \in K \\
0 & x \notin K
\end{cases}
$$

ed essendo $f_1, f_2$ calcolabili, anche $f = \mathcal{X}_K$ è calcolabile, ma è noto che questa funzione non è calcolabile e quindi in generale $f$ non è calcolabile.

\subsection{Esercizio 2}

Dimostrare che un insieme $A$ è RE se e solo se esiste una funzione $f : \mathbb{N} \rightarrow \mathbb{N}$ calcolabile tale che $A = img(f) = \{ f(x) : x \in \mathbb{N} \}$.

\subsubsection{Soluzione}

Se $A$ è RE, allora la sua funzione $SC_A$ è calcolabile ed è definita solo su $A$.

Per fare in modo che $A$ sia l'immagine di $f$, si può prendere

$$
f(x) = x \cdot SC_A(x)
$$

In questo modo, quando $x \in A$, $SC_A(x) = 1$ e quindi $f(x) = x$ e pertanto $f(x) \in A$.
$f$ è calcolabile in quanto è ottenuta componendo funzioni calcolabili.

Se $A = img(f)$ con $f$ calcolabile. L'idea è quella di definire la funzione semi-caratteristica di $A$ in modo che valga $1$ solo quando viene calcolata su un valore $x$ che viene fornito in output dalla funzione.

Sia $e$ un programma che calcola $f$. La funzione semi-caratteristica di $A$ può essere definita come:

$$
SC_A(x) = \mathbb{1}\bigg( \mu w. S \Big(e, (w)_1, x, (w)_2, \Big) \bigg)
$$

\`E necessario effettuare la minimalizzazione perché $f$ può non essere totale, ovvero \textbf{non} può essere definita come:

$$
SC_A(x) = \mathbb{1}\bigg( \mu w. |f(w) - x)| \bigg)
$$

\subsection{Esercizio 3}

$$
A = \{ x | x \in W_x \wedge \phi_x(x) > x \}
$$

\subsubsection{Soluzione}

L'insieme è molto simile a $K$, quindi probabilmente è RE:

$$
SC_A(x) = \mathbb{1} \bigg(\mu w .  x+1 \dotminus \phi_x(x) \bigg) = \mathbb{1} \bigg(\mu w .  x+1 \dotminus \Phi_u(x,x) \bigg)
$$

La funzione è calcolabile, quindi $A$ è RE e va bene anche se $x \notin W_x$ perché la funzione è correttamente indefinita.

Per dimostrare che $A$ non è ricorsivo posso fare la riduzione $K \leq_m A$.

Serve quindi una funzione $g(x,y)$ tale che vista come funzione di $y$ appartenga ad $A$ solo se $x \in K$.

$$
g(x,y) = \begin{cases}
y+1 &x \in K \\
\uparrow &x \notin K
\end{cases} = (y+1) \cdot SC_K(x)
$$

Quando $x$ in $K$, $g(x,x) = x+1$ quindi la condizione di appartenenza ad $A$ è soddisfatta.

Per SMN esiste $f$ che funziona come funzione di riduzione e tale che $g(x,y) = \phi_{f(x)}(y)$:

\begin{itemize}
	\item $x \in K$: $\phi_{f(x)}(y) = g(x,y) = y+1 \forall y$ in particolare, $\phi_{f(x)}(f(x)) = f(x) +1 \forall f(x)$ ed inoltre $f(x) \in W_{f(x)}$, quindi $f(x) \in A$.
	\item $x \notin K$: $\phi_{f(x)} = g(x,y) = \uparrow \forall y$, quindi $f(x) \notin W_{f(x)}$ e pertanto $f(x) \notin A$.
\end{itemize}

$A$ non è ricorsivo ed è RE, quindi $\overline{A}$ è non RE.

\subsection{Esercizio 4}

$$
B = \{ x \in \mathbb{N} | \forall y \in W_x \exists z \in W_x \: (y < z) \wedge \phi_x(y) > \phi_x(z) \}
$$

\subsubsection{Soluzione}

L'insieme contiene tutti gli indici, tali che per ogni valore del dominio, esiste un valore più grande tale che il valore della funzione calcolato in quel punto sia più basso.

L'insieme è saturo, perché

$$
\beta = \{ f \in \mathcal{C} : \forall y \in dom(f). \exists z \in dom(f). y < z \vee f(y) > f(z) \}
$$

C'è una quantificazione universale, quindi è probabile che sia non RE.

Ad occhio c'è una sola funzione che appartiene a $\beta$ è ed $\emptyset$ perché è sempre indefinita.
Dato che $\emptyset$ è una parte finita di tutte le funzioni, basta trovarne una che non appartiene a $\beta$.
Per non appartenere a $\beta$ basta che una funzione non sia decrescente, quindi la funzione $id$ non appartiene a $\beta$ ma ammette parte finta che ci appartiene e quindi per Rice Shapiro, B è non RE.

$$
\overline{\beta} = \{ f \in \mathcal{C} : \exists y \in dom(f). \forall z \in dom(f). y > z \vee f(y) \leq f(z) \}
$$

Osservando $\overline{\beta}$ a prima vista sembra essere non RE, però ragionando meglio sulle condizioni di appartenenza a $\overline{\beta}$ si può notare che viene richiesto che la funzione abbia un minimo e questo è sempre vero per le funzioni definite sui numeri naturali.

Quindi $f \in \overline{\beta} \Leftrightarrow f \neq \emptyset$, ovvero $\overline{\beta} = \mathcal{C} \setminus \{ \emptyset \}$, $\beta = \{\emptyset\}$.

Si può definire la funzione semi-caratteristica:

$$
SC_{\overline{B}}(x) = \mathbb{1}\bigg(  \mu w. H\Big( x, (w)_1, (w)_2\Big) \bigg)
$$ 

Quindi $\overline{B}$ è RE e non ricorsivo, perché $B$ è non-RE.

\subsection{Esercizio 5}

Enunciare il secondo teorema di ricorsione. Utilizzarlo per dimostrare che esiste un indice $e \in \mathbb{N}$ tale che $W_e = \{e^n \: |\: n \in \mathbb{N} \}$.

\subsubsection{Soluzione}

Il secondo teorema dice che, data una funzione $f : \mathbb{N} \rightarrow \mathbb{N}$ calcolabile e totale, $\exists e \in \mathbb{N}$ tale che $\phi_e = \phi_{f(e)}$.

Definisco quindi una funzione che vista come funzione di $y$ abbia le caratteristiche desiderate:

$$
g(x,y) = \begin{cases}
\log_x y &\text{se } y = x^n \text{ per qualche }n \\
\uparrow
\end{cases} = \mu n . |x^n - y |
$$

Essendo $g$ calcolabile, per SMN esiste $f$ calcolabile e totale, tale che $\phi_{f(x)}(y) = g(x,y)$. 

Quindi $W_{f(x)} = \{ x^n \: | \: n \in \mathbb{N} \}$. Per il secondo teorema di ricorsione $\exists e$ tale che $\phi_e = \phi_{f(e)}$.

