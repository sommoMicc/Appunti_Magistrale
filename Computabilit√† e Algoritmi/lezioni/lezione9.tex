% !TEX encoding = UTF-8
% !TEX TS-program = pdflatex
% !TEX root = computabilità e algoritmi.tex
% !TEX spellcheck = it-IT
\section{Composizione generalizzata}\label{composizione-generalizzata}

$$f: \mathbb{N}^k \rightarrow \mathbb{N},\: g_1,\ldots g_n : \mathbb{N}^k \rightarrow \mathbb{N}$$

La loro composizione $h: \mathbb{N}^k \rightarrow \mathbb{N}$ è data da: 

$$
h(\vec{x}) = f(g_1(\vec{x}), \ldots, g_n(\vec{x}))
$$

La funzione $h(\vec{x})\downarrow$ (è definita) se tutte le
$g_i \downarrow y_i, f(y_1,\ldots,y_n) \downarrow$.

L'approccio utilizzato nella valutazione delle funzioni è quello
\textbf{eager}, ovvero vengono valutati prima tutti i parametri.

Ad esempio: $\underline{0} : \mathbb{N} \rightarrow \mathbb{N},\: \underline{0}(x) = 0$ e $d(x) = \uparrow,\: \underline{0}(d(1))$ è
$\uparrow$ perché prima è necessario valutare i parametri.

\subsection{Calcolabilità della funzione
composta}\label{calcolabilituxe0-della-funzione-composta}

Se $f: \mathbb{N}^n \rightarrow \mathbb{N}, g_1\ldots g_n : \mathbb{N}^k \rightarrow \mathbb{N} \in \mathcal{C}$, allora anche $h: \mathbb{N}^k \rightarrow \mathbb{N}$ è calcolabile in $\mathcal{C}$.

\subsubsection{Dimostrazione}\label{dimostrazione}

Siano $F,G_1, \ldots{}, G_n$ programmi URM in forma normale per
le relative funzioni.

L'input della funzione \emph{h} avrà nei primi \emph{k} registri i
valori di input, è necessario quindi andare a copiarli in una locazione
di memoria che non viene usata dai vari programmi, ovvero dalla
locazione \emph{m+1}, con $m = max\{\rho(F), \rho(G_1), \ldots \rho(G_N), k,n\}$.

I risultati parziali dei programmi vengono poi memorizzati a partire
dalla locazione \emph{m + k +1} per poi essere utilizzati da \emph{F}

Il programma risultante è:

\begin{lstlisting}[language=URM]
T([1 ... k],[m+1 ... m+k])
G1[m+1 ... m+k -> m+k+1]
...
Gn[m+1 ... m+k -> m+k+n]
F[m+k+1 ... m+k+n -> 1]
\end{lstlisting}

\subsubsection{Esempio - Somma di due numeri}\label{esempio}

A partire dalla funzione $sum(x_1, x_2) = x_1+x_2$ è possibile andare a ottenere la funzione

$$f(x_1, x_2, x_3) = x_1 + x_2 +x_3$$

componendo la funzione \textit{sum} con se stessa:

$$f(x_1, x_2, x_3) = sum(sum(x_1,x_2),x_3)$$

Tuttavia, strettamente parlando, le $g_i$ non hanno la stessa
arietà, pertanto sono necessari dei piccoli aggiustamenti:

$$f(x_1, x_2, x_3) = sum(sum(U_1^{3}(\vec{x}),U_2^3(\vec{x}))), U_3^3(\vec{x}))$$

\section{Ricorsione Primitiva}\label{ricorsione-primitiva}

\begin{align*}
	fact(0) &= 1 \\
	fact(n+1) &= (n+1)fact(n)
\end{align*}

\begin{align*}
	fib(0) &= 1 \\
	fib(1) &= 1 \\
	fib(n+2) &= fib(n+1) + fib(n)
\end{align*}

Date $f: \mathbb{N}^k \rightarrow \mathbb{N}$ e $g:\mathbb{N}^{k+2} \rightarrow \mathbb{N}$, la funzione per ricorsione primitiva $h: \mathbb{N}^{k+1} \rightarrow \mathbb{N}$ è definita come

\begin{align*}
	h(\vec{x}, 0) &= f(x) \\
	h(\vec{x}, y+1) &= g(\vec{x}, y, h(\vec{x},y))
\end{align*}

Così facendo viene definita \emph{h} utilizzando \emph{h} e
concettualmente è corretto, tuttavia è necessario dimostrare formalmente
l'esistenza e l'unicità di \emph{h}.

Questo lo si fa considerando l'operatore $\Phi$:

\begin{align*}
	\Phi : (\mathbb{N}^k \rightarrow \mathbb{N}) &\rightarrow (\mathbb{N}^k \rightarrow \mathbb{N}) \\
	\Phi(h)(\vec{x}, 0) &= f(\vec{x}) \\
	\Phi(h)(\vec{x}, y+1) &= g(\vec{x}, y, h(\vec{x},y))
\end{align*}

tale che $\Phi(h) = h$, ovvero tale che viene raggiunto un punto
fisso. Ciò sarebbe da dimostrare, ma questo va oltre l'obiettivo del corso, quindi viene dato per buono.

In altre parole, se \emph{h} rispetta lo schema precedentemente definito, allora esiste ed è unica.

Assumendo quindi che la ricorsione primitiva di una funzione calcolabili
è calcolabile, è possibile definire varie operazioni senza scrivere
esplicitamente il programma per calcolarle:

Ad esempio la funzione somma può essere definita in modo ricorsivo:

$$
	x+y = h(x,y) =\begin{cases}
	x+0 = x &\Rightarrow f(x) = x \\
	x+(y+1) = (x+y) + 1 &\Rightarrow g(x,y,z) = succ(z)
	\end{cases}
$$

In modo simile possono essere anche definiti il prodotto e l'esponenziale:

$$
x \cdot y = h(x,y) =\begin{cases}
x \cdot 0 = 0 &\Rightarrow f(x) = 0 \\
x \cdot (y+1) = (x \cdot y) + x &\Rightarrow g(x,y,z) = z+x
\end{cases}
$$

$$
x^y = h(x,y) =\begin{cases}
x^0 = 1 &\Rightarrow f(x) = 1 \\
x^{(y+1)} = (x^y) \cdot x &\Rightarrow g(x,y,z) = z \cdot x
\end{cases}
$$

\subsection{Calcolabilità della Ricorsione Primitiva}\label{calcolabilituxe0-della-ricorsione-primitiva}

Siano $f:\mathbb{N}^k \rightarrow \mathbb{N}$ e $g:\mathbb{N}^{k+2} \rightarrow \mathbb{N} \in \mathcal{C}$ allora anche \emph{h} definita per
ricorsione primitiva è in $\mathcal{C}$. Ovvero quello che è stato assunto precedentemente.

\subsubsection{Dimostrazione}\label{dimostrazione-1}

Siano \emph{F} e \emph{G} i programmi che calcolano \emph{f} e \emph{g}.

Il programma che calcola \emph{h} verrà invocato con il vettore \emph{x}
nelle prime \emph{k} locazioni e con \emph{y} nella locazione
\emph{k+1}.

Come prima cosa è necessario trasferire i dati di input in una zona di
memoria non utilizzata dai programmi \emph{F} e \emph{G}, ovvero
\emph{m+1}, dove $m = max\{\rho(F), \rho(G), k+2\}$.

Dopodiché è necessario effettuare un'interazione su un contatore
\emph{i} che parte da 0, fino a quando non viene raggiunto \emph{y}.

\begin{verbatim}
h(vec(x), 0) = f(vec(x)) -- i=0 -- i==y? NO
h(vec(x), 1) = g(vec(x), 0, h(vec(x),0)) -- i=1 -- i==y? NO
...
h(vec(x), i+1) = g(vec(x), i, h(vec(x),i)) -- i=n -- i==y? SI -> fine
\end{verbatim}

ovvero il programma per \emph{h} sarà:

\begin{lstlisting}[language=URM]
T([1..k], [m+1 ... m+k]) // copia x
T(k+1, m+k+3) //copia y
F[m+1 ... m+k -> m+k+2]
J(m+k+3, m+k+1, END) #LOOP
G[m+1 ... m+k+2 -> m+k+2] // h(vec(x),i+1)
S(m+k+1) //i++
J(1,1,LOOP)
T(m+k+2,1) #END
\end{lstlisting}

Se \emph{f} e \emph{g} sono funzioni parziali quanto dimostrato richiede
maggiori precisazioni, ma a noi basta sapere che se durante il calcolo
troviamo qualche funzione non definita, anche \emph{h} non è definita.

\subsection{Funzioni totali definite ricorsivamente}\label{osservazione-senza-titolo}

Le funzioni definite a parte da funzioni totali mediante composizione o
ricorsione sono anche loro totali.

La dimostrazione per le funzioni mediante composizione è vera per
definizione, l'altra è lasciata per esercizio (si fa per induzione).

\todo[inline]{todo}

\subsection{Esercizio - Libreria di funzioni calcolabili}\label{esercizio---libreria-di-funzioni-calcolabili}

\begin{itemize}
\item
  somma \emph{x+y}
\item
  prodotto $x \cdot y$
\item
  esponenziale $x^y$
\item
  fattoriale \emph{fact(x)}
\end{itemize}

Sguardo d'insieme: vogliamo trovare un programma URM in grado di
simulare un altro programma URM, le operazioni numeriche sono
interessanti perché il programma in input verrà rappresentato come un
numero.

\subsubsection{Predecessore}\label{predecessore}

$$ \text{pred}(x) = \begin{cases}x \dotminus 1 = 0, \:& \text{ se } x=0\\
x-1, \:& \text{ se } x > 0\end{cases}$$

Può essere calcolata ricorsivamente con

\begin{align*}
	0 \dotminus 1 &= 0 \\
	(y+1) \dotminus 1 &= y
\end{align*}

\subsubsection{Sottrazione tra numeri naturali}\label{sottrazione-tra-numeri-naturali}

$$x \dotminus y =\begin{cases}
0,\:& \text{se } x \leq y\\
x-y, \:& \text{altrimenti}
\end{cases}$$

Può essere calcolata ricorsivamente con:

\begin{align*}
x \dotminus 0 &= x \\
x\dotminus (y+1) &= (x \dotminus y) \dotminus 1
\end{align*}

\subsubsection{Segno}\label{segno}

$$sg(x) =\begin{cases}
0,\:& \text{se } x = 0\\
1, \:& \text{altrimenti}
\end{cases}$$

Può essere calcolata ricorsivamente con:

\begin{align*}
sg(0) &= 0 \\
sg(y+1) &= 1
\end{align*}

In modo simile può essere definito anche $\overline{sg}$.

\subsubsection{Valore assoluto della differenza}\label{valore-assoluto-della-differenza}

$$|x - y|=\begin{cases}
x-y,\:& \text{se } x \leq y\\
y-x, \:& \text{altrimenti}
\end{cases}$$

Può essere calcolata in modo composizionale con:

$$|x - y| = (x \dotminus y) + (y \dotminus x)$$


\subsubsection{Minimo}\label{minimo}

$$ \min (x,y) =\begin{cases}
x,\:& \text{se } x \leq y\\
y, \:& \text{altrimenti}
\end{cases}$$

Può essere calcolata con:

$$\min(x,y) = x \dotminus (x \dotminus y)$$

Questo perché se \emph{x} è il minimo, $x\dotminus y$ è 0.

\subsubsection{Resto della divisione intera}\label{resto-della-divisione-intera}

$$\text{rm}(x,y) =\begin{cases}
y \mod x,\:& \text{se } x \neq y\\
y, \:& \text{altrimenti}
\end{cases}$$


Può essere calcolato ricorsivamente come:

\begin{align*}
\text{rm}(x,0) &= 0 \\
\text{rm}(x, y+1) &= \begin{cases}\text{rm}(x+y) +1, \:& \text{ se rm}(x+y) +1 \neq x\\
0, \:& \text{ altrimenti} \end{cases}
\end{align*}

L'\emph{if} può essere espresso in modo algebrico utilizzando la
funzione \emph{sg}, ottenendo:

$$\text{rm}(x, y+1) = (sg(x \dotminus \text{rm}(x,y) \dotminus 1))(\text{rm}(x,y)+1)$$

\subsubsection{Esercizio - Divisione intera}\label{esercizio---divisione-intera}

$$\text{qt}(x,y) =\begin{cases}
\floor[\Big]{\frac{y}{x}},\:& \text{se } x \neq y\\
y, \:& \text{altrimenti}
\end{cases}$$

\todo[inline]{todo}

\subsubsection{Esercizio - Definizione per casi}\label{esercizio---definzione-per-casi}

$f_1 \ldots f_n : \mathbb{N}^k \rightarrow \mathbb{N}$ totali e calcolabili e
$Q_1, \ldots, Q_n \subseteq \mathbb{N}^k$ decidibili e tali che per ogni $\vec{x}$ solo un predicato è vero.

Definire $f(x) = f_1(x) \text{ se } Q_1(x) \ldots f_n(x) \text{ se } Q_n(x)$

\todo[inline]{todo}
